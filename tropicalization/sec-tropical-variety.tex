\section{Tropical Variety}
\label{sec:tropical-variety}
	Let $K[x_1^{\pm1}, \ldots, x_n^{\pm1}]$ 
	denote the ring of \textbf{Laurent polynomials} over $K$.
	Thus an arbitrary 
	Laurent polynomial $f \in K[x_1^{\pm1}, \ldots, x_n^{\pm1}]$ is 
	in the form 
	\[
	f = \sum\limits_{\mathbf{u} \in \Z^n} c_{\mathbf{u}}x^{\mathbf{u}}
	\]
	\begin{definition}[\citet{Maclagan2015}]
		Given a valuation,
		the \textbf{tropicalization} 
		\[
		\trop(f)(w) = \min\limits_{u \in \Z^n} (\val(c_\mathbf{u} + \sum\limits_{i = 1}^{n} u_i w_i)).
		\] 
		is a piecewise, linear, real-valued function
		$f: \R^{n+1} \rightarrow \R$ 
		that is obtained by replacing each coefficients $c_{\mathbf{u}}$ 
		by its valuation and 
		by performing all additions and multiplications 
		in the tropical semiring $(\R, \oplus, \odot)$.
	\end{definition}
	
	The variety of the Laurent polynomial $f \in K[x_1^{\pm 1}, \ldots, x_n^{\pm 1}]$ 
	is a hypersurface in the algebraic torus $T^n$ over the algebraically 
	closed field $K$:
	\[
	V(f) = \{ y \in T^n: f(y) = 0 \}.
	\]
	We can check that in fact $\trop(V(f)) = V(\trop(f))$.
	
	\begin{example}[A Tropical Line]
		We have seen this example in previous chapter.
		Let $f =  x + y + 42$. 
		Then $\trop(f) = \min(x, y, 0)$ since any constant is mapped 
		to $0$ under the canonical valuation.
		Thus we have 
		\[
		\trop(V(f)) = \{x = y \le 0\} \cup \{x = 0 \le y\} \cup \{y = 0 \le x \}.
		\]
		Since this tropical variety has degree of only $1$,
		we call it a \textbf{tropical line}, 
		shown in the next figure. 
		This result is consistent with our result from the tropical 
		plane curve section.
	\end{example}
	
	Now we make an important connection between 
	the polyhedral geometry of tropical hypersurfaces,
	and the tropical varieties. 
	\begin{proposition}
		Let $f \in K[x_1^{\pm 1}, \ldots, x_n^{\pm 1}]$ 
		be a Laurent polynomial.
		The tropical hypersurface $\trop(V(f))$ is the support of
		a pure $\Gamma_{\val}$-rational polyhedral complex 
		of dimension $n - 1 \in \R^n$.
		It is also the $(n -1)$-skeleton of the polyhedral complex 
		dual to a regular subdivision of the Newton polytope of 
		\[
		f = \sum c_\mathbf{u} x^\mathbf{u}
		\]
		given by the weights $\val(c_\mathbf{u})$ on the lattice points
		in the Newton polytope of $f$. 
	\end{proposition}
	This description of tropical varieties plays an important role in geometric tropicalization as we will see in later chapter. 
