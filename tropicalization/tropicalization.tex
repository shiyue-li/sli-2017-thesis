\chapter{Tropicalization}
\label{chp:tropicalization}
\epigraph{
Algebra is the offer made by the devil to the mathematician...All you need to do, is give me your soul: give up geometry
}{Michael Atiyah}

In this chapter,
we will explore one of the central topics in tropical geometry:
trees and their parameter spaces.
We first start from the study of linear spaces,
which set our journey to begin from the study of hyperplane arrangements.
These hyperplane arrangements push us to the combinatorial world,
where we incorporate the theory of matroids, 
borrowed from classical combinatorics and theoretical computer science.
We will see that the tropicalized linear spaces 
can be parameterized by the Grassmannian;
in particular, the Grassmannian $\gr(2, n)$
parameterizes the lines in the projective space $\P^{n-1}$,
and the tropicalization of it 
can be ``identified" with the space of phylogenetic trees 
from computational biology. 
All these set the foundations for us to study the tropicalization of 
a complete intersection, 
which is the subject of intersection theory in later chapter.

We first review some basic constructions in polyhedral geometry in section 1.
Tropical geometry has a strong connection 
with polyhedral geometry,
mostly because the tropical varieties will give us polytope, 
which gives us rich combinatorial structures. 
Polyhedral fans are tightly connected with toric varieties,
which in later chapters will be an important tool for
solving classical algebraic geometry problems. 
In this chapter, we are mainly following the treatment of \citet{Maclagan2015} and various articles cited inline. 



	
	
	
	
	
	
	
		


	

  

