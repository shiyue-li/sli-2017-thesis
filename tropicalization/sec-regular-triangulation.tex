\section{Regular Triangulations}
\label{sec:regular-trangulations}
    We will not dive into the details of triangulations in this manuscript; however, it helps us formally visualize the polyhedral complices induced by tropical varieties. 
	\begin{definition}[Triangulation]
		A \emph{triang
		ulation} $T$ of a point set 
		$\cala = \{a_1,\cdots,a_n\}$ is a set of simplexes 
		such that their vertices are in $\cala$, 
		their union equals the convex hull $\conv(A)$, 
		and the intersection of any pair of simplexes 
		is their common (possibly empty) face.
	\end{definition}
	
	For a fixed triangulation $T$ of $\cala$,
	every vector $\phi \in \R^n$ induces a unique piecewise linear 
	function $g_{\phi, T}(a_i) = \phi_i$ 
	at the vertices of $T$
	and by the requirement that 
	$g_{\phi, T}(a_i)$ should be an affine function 
	(meaning a linear function plus some constant
	such that it looks like a straight line in a plane
	or just linear space translated in higher dimensions)
	on each simplex of $T$.
	
	Now we give a slightly less technical definition of the regular 
	triangulations.
	\begin{definition}[Regular Triangulations]
		A triangulation $T$ of $\cala$ is said to be \textbf{regular}
		if there exists a vector $\phi \in \R^n$ 
		such that $g_{\phi, T}$ is strictly convex over $T$.
	\end{definition}
	
	From this definition,
	we can already see how these polyhedral complices are 
	connected to tropical geometry
	via this simple relationship between 
	the regular triangulation in $\R^d$ 
	and the convex hulls in $\R^{d+1}$.
	By having this piecewise linear function 
	$\phi: \cala \rightarrow \R$,
	we are simply lifting every point $a_i \in \cala$ 
	to the graph of $\phi$,
	which form a convex hull above the graph formed by 
	all the vertices in $\cala$. 
	The \emph{lowest} part of this convex hull 
	is thus the graph of the convex piecewise linear function.