%%%% Section 2
\section{Cycles, Rational Equivalence and the Chow Group}
\label{sec:cycles-rational-equivalence-chow-group}
We want to define our most basic algebraic structure here.
	\begin{definition}[Free Abelian Group]
		A group $G$ is a \textbf{free abelian group}
		if 
		\[
		G \cong \bigoplus\limits_{i \in I} \Z
		\]
		for some arbitrary index set $I$. 
	\end{definition}
	A free abelian group has an associated basis $\calb$ if and only if
	$\calb$ generates $G$ and for $b_1, \ldots, b_n \in \calb$ 
	and $c_1, \ldots, c_n \in \Z$,  
	\[
	\sum\limits_{i=1}^n c_i b_i = 0,
	\]
	implies that $c_1 = \cdots = c_n = 0$. 
	

	\begin{definition}[Group of Cycles]
		Let $X$ be any algebraic variety. 
		The \emph{group of cycles} on $X$, denoted $Z(X)$,
		is the free abelian group generated by the set of subvarieties
		(or reduced irreducible subschemes) of $X$.
		
		The group $Z(X)$ is graded by dimension:
		we write $Z_k(X)$ for the group of cycles 
		that are formal linear combinations of 
		subvarieties of dimension $k$
		(these subvarieties are also called $k$-cycles),
		so that we have
		\[
		Z(X) = \bigoplus_k Z_k(X).
		\]
	\end{definition}

For subvarieties $Y_i \in X$ with some arbitrary index set $I$,
we call a cycle 
	\[
	Z = \sum\limits_{i \in I} n_i Y_i
	\] is \textbf{effective} if all the coefficients $n_i$ are nonnegative. 
In other words, a cycle on an arbitrary algebraic variety (or scheme) $X$
is a finite formal sum of (irreducible) subvarieties of $X$,
with integer coefficients.
For a given field $K$, we say that \textbf{rational functions}
are algebraic fractions with both numerators and denominators 
being polynomials.
U 

Another important definition that will appear many times in the future is 
	\begin{definition}[(Weil) Divisor]
		A \textbf{(Weil) divisor} is an $(n-1)$-cycle on a pure $n$-dimensional 
		scheme. 
	\end{definition}
How can we intuitively see divisor and why is it called divisor? 
Loosely speaking, divisors are integer linear combinations of 
codimension-$1$ subvarieties.
Let us think about an easy case ($1$-dimensioanl case): 
Given two plane curves and let them intersect,
the integer combination of the intersection points (possibly with multiplicity) 
is a divisor. 

	\begin{definition}[Rational Equivalence]
		We say that two $i$-cycles $Z_1$ and $Z_2$ 
		are rationally equivalent if and only if 
		there exists a subvariety $V \subset \P^1 \times X$ 
		and two points $x_1, x_2 \in \P$ such that
		\[
		(V \cap (\{x_1\} \times X)) - (V \cap (\{x_2\} \times X)) = 0.
		\]
		
		We say that two subschemes are rationally equivalent if 
		their associated cycles are rationally equivalent.
	\end{definition}

	\begin{definition}[Chow Group]
		Let $\rat(X)$ be the set of rational equivalence classes 
		of cycles of $X$.
		The \textbf{Chow Group} of $X$ is the quotient
		\[
		A(X) = Z(X)/\rat(X),
		\]
		or in other words,
		the \textbf{group of rational equivalence classes of cycles 
		on $X$}. 
		If $Y \in Z[X]$ is a cycle, we write $[Y] \in A(X)$ 
		for its equivalence class;
		if $Y$ is a subscheme, we write $[Y]$ 
		as the class of the cycle $\inner{Y}$ associated to $Y$. 
	\end{definition}

We mentioned previously that Chow groups are graded by dimension,
and let us restate the following.
	\begin{theorem}
		If $X$ is a scheme then the Chow group of $X$ 
		is graded by dimensions;
		that is 
		\[
		A(X) = \bigoplus A_k(X)
		\]
		where $A_k(X)$ is the group of rational equivalence classes of
		$k$-cycles. 
	\end{theorem}

If we see through what B\'{e}zout's Theorem says,
we see that for plane curve $A$ and $B$, they at most intersect at 
$(\deg A)(\deg B)$ points with multiplicity. 
This implies that when we count the subvarieties that lie at the intersection,
there should be some ``product structure" analogous to "cup product" 
of cohomology ring that give us some ring structure.

Let us state this general theorem.
	\begin{theorem}[Unique Product Structure on $A(X)$]
		If $X$ is a smooth quasi-projective variety,
		then there is a unique product structure on $A(X)$ 
		satisfying the condition:
		If two subvarieties $A, B$ of $X$ are generally transverse;
		that is, they meet transversely at a general point of each component of $C$ of $A \cap B$,
		then we have
		\[
		[A][B] = [A \cap B].
		\]
		This structure makes the subvariety $A(X)$ 
		\[
		A(X) = \bigoplus\limits_{c = 0}^{\dim X} A^c(X)
		\]
		into an associative, commutative ring, 
		graded by codimension, called the Chow ring of $X$. 
	\end{theorem}
	The products rational equivalent cycles are in fact 
	rationally equivalent as well (see \citet{Fulton1993}). 