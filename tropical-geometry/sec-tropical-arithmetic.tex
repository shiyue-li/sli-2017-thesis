\section{Tropical Arithmetic, Valuations and Tropical Plane Curves}
\label{sec:tropical-arithmetic-valuations-tropical-plane-curves}
    Tropical plane curves are degeneration of algebraic curves through 
    tropical geometry. 
    By degenerating smooth curves in the limits, we obtain singular curves
    with many irreducible components and rich combinatorial structure. 
    In 1971, the work of \citet{Bergman1971} started such discussion on singular curves,
    which only in hindsight did people realize this sets the foundation for 
    tropical geometry. 
    Later on, people became interested in families of such tropical curves,
    which sit in the intersection of classical algebraic geometry 
    and tropical geometry: 
    the \emph{tropical moduli space of curves} 
    and its compactification problems
    (in particular Deligne-Mumford compactification by stable curves).
    We will gradually see the connection as we move along.
    Here we will follow this historical line 
    for our study of tropical plane curves,
    moving from studying plane curves 
    to studing these curves abstractly,
    without referring to particular embeddings 
    in the affine or projective space.
    
    Let $K$ be a field and $K^\ast = K \setminus \{0\}$.
    The natural setting for tropical geometry is over 
    \emph{nonarchimedean fields}. 
    \begin{definition}[Nonarchimedean Valuation]
    	A \textbf{nonarchimedean valuation } on $K$ is a map 
	$v: K^\ast \rightarrow \R$ that satisfies:
	\begin{enumerate}
		\item[(1)]
		$v(ab) = v(a) + v(b)$ and 
		\item[(2)]
		$v(a+b) \ge \min(v(a), v(b))$.
	\end{enumerate}
	for all $a, b \in K^\ast$. 
	By convention, we may extend $v$ to $K$ by setting $v(0) = \infty$.
    \end{definition}
    The ring $R \subseteq K$ of elements with nonnegative valuation 
    is the \emph{valuation ring} of $K$.
    The readers may check that $R$ is a local ring and 
    we let $k = R/\mathfrak{m}$ for its residue field. 
    
    \begin{remark}
    	The valuation is called nonarchimedean because 
	the axiom of Archimedes do not hold on the image of the map.
	The axiom of Archimedes states that 
	for an ordered field $\F$ (in our case $\R$),
	if for $x, y \in \F$ and $x, y > 0$, there exists $n \in \N$ such that 
	$nx > y$. 
	This axiom allows infinitely large and infinitely small elements to exist
	in the field. 
    \end{remark}
    
    Let us look at an example that explores some property of the valuation. 
    \begin{example}
    	Let $a, b \in K$. If $v(a) \ne v(b)$, then
	\[
	v(a + b) = \min(v(a), v(b)).
	\]
	In other words, 
	for all $a, b \in K$,
	the minimum of $v(a), v(b), v(a + b)$ occurs at least twice
	amongst the three.

	We first discover following properties:
	\begin{enumerate}
		\item[(1)] $v(1) = 0$. For any $x \in K$, $v(x) = v(x \cdot 1) = v(x) + v(1) \Leftrightarrow v(1) = 0$. 
		\item[(2)] $v(-1) = v(1) = 0$. Since $v(1) = v(-1) \cdot v(-1) = 2v(-1) = 0 \Leftrightarrow v(-1) = v(1) = 0$.
		\item[(3)] $v(-x) = v(x)$ for all $x \in K$, since $v(-x) = v(-1 \cdot x) = v(-1) + v(x) = v(x)$.
	\end{enumerate}
	Without loss of generality, we may assume $v(a) < v(b)$.
	Then we have 
	\[
	v(a) = v(a + b - b) = v((a+b) - b) \ge \min(v(a+ b), v(-b)) = \min(v(a+b), v(b)).
	\]
	However, by assumption $v(a) < v(b)$ so $v(a)$ can only be greater or equal to $v(a+b)$. 
	Thus $v(a) \ge v(a+b) \ge \min(v(a), v(b)) = v(a)$.
	Therefore, $v(a + b) = v(a)$, the smaller of the two. 
    \end{example}
    
    \begin{example}[A Tropical Line]
    	Let $f(x, y) = x + y - 42$ and let $X = V(f)$. 
	    We can see that $X$ is $\P^1$ minus $3$ points (why?) 
	    and what is $\trop(X)$? 
	
	\begin{figure}
	\begin{center}
	\begin{tikzpicture}[
	axis/.style={thin, ->, >=stealth'}]
	    % Draw coordinates
		\draw[axis] (0, 0) -- (1, 0) node[right]{$\mathbf{e}_1$};
		\draw[axis] (0, 0) -- (0, 1)
		node[above]{$\mathbf{e}_2$};
		\draw[axis] (0, 0) -- (-1, -1) node[below]{$-\mathbf{e}_1 - \mathbf{e}_2$};
	\end{tikzpicture}
	\end{center}
	\caption{A tropical line arisen from the polynomial $f(x, y) = x + y - 42$.}
	\label{fig:tropical-line}
    \end{figure}
	
	We want to find $x$ and $y$ such that $x + y - 42 = 0$.
	By the previous example, we know that 
	the minimum of $v(x), v(y),$ and $v(42)$ is attained at least twice.
	That is the tropical curve is some creature living in $\R^2$ such that 
	the minimum of $z = v(x), w = v(y)$ and $0 = v(42)$ is attained
	at least twice:
	\[
	\trop(X) = \{(z, w) \in \R^2: \text{ the minimum of $z, w, 0$ is attained at least twice}\}.
	\]
	In Figure \ref{fig:tropical-line}, we see that it is three rays from the origin 
	along the direction of $\mathbf{e}_1, \mathbf{e}_2$ and $-\mathbf{e}_1-\mathbf{e}_2$ 
	for standard basis vector $\mathbf{e}_1, \mathbf{e}_2$. 
    \end{example}
   
    A more generalized theorem of the previous two examples 
    can be stated as a ``plane"-version of the Kapranov's Theorem:
    \begin{theorem}
    	Let $f: \K^2 \rightarrow \R$ be defined by
	    \[
	    f = \sum\limits_{(i, j) \in \Z^2} c_{ij} x^i y^j \in K[x^{\pm}, y^{\pm}].
	    \]
	    Then 
	    \[
	    \trop(X) = \{(z, w) \in \K^2: \min\limits_{(i, j) \in \Z^2} v(c_{ij}) + iz + jw \text{ is attained at least twice}\}.
	    \]
    \end{theorem}
    Tropical Geometry is based on a newly defined mathematical subject: 
    tropical semiring $(\R \cup \{\infty\}, \oplus, \odot)$.
    The two binary operations tropical addition $\oplus$ and tropical multiplication $\otimes$ are defined as follows:
    For $a, b \in \R \cup \{\infty\}$
    \[
    a \oplus b = \min{a, b},
    a \otimes b = a + b
    \] where the $+$ on the right hand side is conventional addition. 
    
    Readers can easily check that both distributivity and associativity hold for both the tropical addition and tropical multiplication.
    In addition, we can identify $\infty$ as the additive identity and $0$ as the multiplicative identity. 
    Therefore, we have obtain this tropical semiring as foundation of our study.
    
    Based on tropical arithmetic, 
    we can define tropical polynomials and define tropical varieties similarly as in classical algebraic geometry.
    Many important theorems such as B\'{e}zout Theorem in classical algebraic geometry hold. 
    The process of
    compactifying of a space, 
    which is needed in many techniques in algebraic geometry such as intersection theory,
    can also find a nice tropical version of it.
    In this thesis, 
    our center of focus is going to be studying compactification of moduli spaces called Hassett spaces, 
    under tropical setting.