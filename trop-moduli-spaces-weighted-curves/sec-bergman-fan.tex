\section{Bergman Fan and Nested Sets}
\label{sec:bergman-fan-and-nested-sets}
	We now describe a fan that is a generalization of 
	the simplicial fan generated from the lattice $\call(B)$ in Section \ref{sec:hyperplane-arrangement}.
	\begin{theorem}
		Let $M$ be a matroid on $E = [n]$.
		The collection of cones, 
		who is a positive span of the standard basis 
		associated with the all chains of flats of $M$,
		forms a pure simplicial fan of dimension $\rho(M) - 1$
		in $K^{n+1}/K$.
		The support of this fan equals the tropical linear space
		$\trop(M)$.	
	\end{theorem}
	The dimension of the simplicial fan can be explained as:
	since each chain of flats of the matroid $M$ can be extended to 
	a maximal chain, 
	and each maximal chain of flats $\rho(M) p- 1$ proper flats.
	
	In general, 
	$\trop(M)$ can be identified with $\Delta_M$, the order complex of 
	the geometric lattice of $M$.
	There is a coarser fan structure associated with $\trop(M)$ 
	give by the order complex as well, called \textbf{Bergman fan}.
	We describe the combinatorial description of this Bergman fan,
	since this aspect is the most concrete and computable aspect to us. 
	
    %% Subsection
    \subsection{Bergman Fan}
    \label{subsec:bergman-fan}
        \begin{definition}[Bergman Fan]
        \label{def:bergman-fan}
            For any matroid $M$ with ground set $E(M)$
            we associate a polyhedral fan, 
		    the Bergman fan $B(M) \subseteq \R^{|E|}$ 
		    in the following manner:
            \[
            B(M) \coloneqq \{w \in \R^{|E|}; M_w \text{is loop-free}\}
            \] 
            where $M_w$ is the matroid on $E$ 
            whose bases are all bases $B$ of $M$ of minimal 
		    $w$-weight $\sum_{i \in B}w_i$. 
        \end{definition} 

	    \begin{definition}[Equivalent definition of Bergman fan]
	    \label{def:bergman-fan-2}
    	    For a matroid $M$ on the ground set $[n] = \{1, 2, \ldots, n\}$,
    	    the Bergman fan $B(M) = \{w \in \mathbb{R}^n\}$ 
    	    such that for every circuit $C \in M$,
    	    the minimum of the set $\{w_i | i \in C\}$ is attained 
		    at least twice.
        \end{definition}
    
    %% Subsection
    \subsection{Nested Sets}
    \label{subsec:nested-sets}
        Work of \citet{Feichtner2005} and \citet{Maclagan2015},
        shows that there are several polyhedral structures constructed on the Bergman fan using the theory of building set.
        Recall that we introduced the definition of lattice of flats before (see Definition \ref{def:lattice-of-flats}
        and \ref{def:flat-of-a-matroid}).
        A building set is defined as 
        \begin{definition}[Definition 2.3, \citet{Cavalieri2014}]
        \label{def:building-set}
            Let $\calf$ be tha lattice of flats of a matroid $M$.
            For two flats $F, F' \in \calf$ we write $[F, F'] = \{G \in \calf: F \subseteq G \subseteq F'\}$. 
            A building set for $\calf$ is a subset $\calg$ of $\calf\setminus \{\varnothing\}$ such that the following holds:
            For any $F \in \calf\setminus \{\varnothing\}$, 
            let $\{G_1, \ldots, G_k\}$ be the maximal elements of $\calg$ contaiend in $F$.
            Then there is an isomorphism of partially ordered sets:
            \[
            \varphi_F: \prod\limits_{j=1}^{k} [\varnothing, G_j] \rightarrow [\varnothing, F],
            \] where the $j$-th component of $\varphi_F$ is the inclusion 
            $[\varnothing, F]$.
        \end{definition}
        
        \begin{definition}
        \label{def:nested-set}
            A subset $\cals$ of a building set $\calg$ is called \textbf{nested}, 
            if for any of incomparable elements $F_1, \ldots, F_l$ in $\cals$ with $l \ge 2$,
            the join of all the $F_i$ is not an element of $\calg$.
        \end{definition}
        \citet{Cavalieri2014} tells us that the nested sets of $\calg$ can form an abstract simplicial complex where a subset of a nested set is a nested set, coinciding with the definition of independent set of a matroid).
        By assigning each of the flat $F$ a vector $v_F \in \R^{[E]}$ 
        where $E$ is the ground set of the matrid,
        and accordingly a cone for each nested set of $\calg$,
        we obtain a polyhedral fan whose support (Definition \ref{def:support-of-a-polyhedral-complex}) is $B(M)$ where $M$ is the matroid. 
    
   	    \begin{definition}
    	\label{def:bergman-fan-quotient-space}
    	    Since for any matroid $M$,
    	    the Bergman fan contains the linear space $L$,
    	    spanned by the vector $(1, 1, \ldots, 1)$,
    	    we study the quotient space $B'(M) = B(M)/L$.
        \end{definition}
        

    

    

